\section{Introduction}
% problem statement
% motivation
% general approach
% (outlook on results)

%conventions: lower-scalar, lower,bold-vector, upper,bold-matrix

Finding periodic orbits in dynamic systems has applications in engineering. %cite
This project's goal is the implementation and evaluation of methods for this task.
This document aims at explaining the methods, techniques and ideas used in a concise and connected manner.

The general problem can be described as follows.
Given a system of differential equations $\frac{d \mathbf x}{dt} = \mathbf f(\mathbf x)$, find solutions $\mathbf y$ such that $\mathbf y(t+T) = \mathbf y(t)$ for all $t \in \R$.
Finding closed-form representations of these solutions is impossible in general.
We thus resort to numerical methods for the task.

The tool used for the task of judging whether a given solution is a correct periodic solution is \emph{Galerkin}'s method.
This is a form of collocation method, allowing to convert a continuous operator equation to a system of equations.
Solving this possibly non-linear system yields a periodic solution.

Given a periodic solution for a system smoothly depending on a parameter, if the parameter varies only slightly, so will the solution.
This can be used to track solutions over a broad range of the parameter.
Projecting the found periodic orbits to a simpler object (for example a set of scalars) allows to plot them against the parameter in a bifurcation diagram.
This allows to evaluate the global behavior of the system.
Being able to dynamically create bifurcation diagrams for a given system is a goal of this project.

