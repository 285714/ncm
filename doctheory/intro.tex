\section{Introduction}
% problem statement
% motivation
% general approach
% (outlook on results)

%conventions: lower-scalar, lower,bold-vector, upper,bold-matrix

Finding periodic orbits in dynamic systems has various applications mainly in engineering. %cite
This project's goal is the implementation and evaluation of methods for this task.
This document aims at explaining the methods, techniques and ideas used in a concise and connected manner.

The general problem can be described as follows.
Given a system of differential equations $\frac{d \mathbf x}{dt} = \mathbf f(\mathbf x)$, find solutions $\mathbf y$ such that $\mathbf y(t+T) = \mathbf y(t)$ for all $t \in \R$.
Finding closed-form representations of these solutions is impossible in general.
We thus resort to numerical methods for the task.

The tool used for the task of judging whether a given solution is a correct periodic solution is \emph{Galerkin}'s method.
This is a form of collocation method, allowing to convert a continuous operator equation to a system of equations.
Solving this possibly non-linear system yields a periodic solution.

Given a periodic solution for a system smoothly depending on a parameter, if the parameter varies only slightly, so will the solution.
This can be used to track solutions over a broad range of the parameter.
Projecting the found periodic orbits to a simpler object (for example a set of scalars) allows to plot them against the parameter in a bifurcation diagram.
This allows to evaluate the global behavior of the system.
Being able to dynamically create bifurcation diagrams for a given system is a goal of this project.


\subsection{Notation}

Throughout this text, the following conventions are used
\begin{compactitem}
	\item Scalar values are represented by lowercase letters ($a$, $\tilde b$), vectors by lowercase, bold letters ($\mathbf v$), matrices and other higher dimensional constructs by capital, bold letters ($\mathbf M$).
	\item Subsets of the natural numbers up to an upper limit are denoted using subscript notation
		\[
			\N_n \coloneqq \{ x \ |\ x \in \N, x \le n\} \text.
		\]
	\item Sequences of natural numbers are notated in square brackets
		\[
			\text{for } a,b \in \N,\ a \le b,\ [a:b] \coloneqq (a+i-1)_{i \in \N_{b-a+1}} \text.
		\]
	\item Indexing of vectors is done using subscript notation, matrices are indexed using subscript for the row and superscript for the column.
	\item Vectors and matrices are build using subscript notation
		\[
			\mathbf M = (f(i,j))_{i,j \in \N_n} \equiv \mathbf M_i^j = f(i,j) \text{ for } i,j \in \N_n \text.
		\]
	\item Function application using curly brackets denotes element wise application of a function.
		That is, for sets $U$,$V$, $\mathbf u \in U^n$ for an $n \in \N$ and a function $f: U \to V$,
		\[
			f\{\mathbf u\} = \begin{pmatrix}
				f(\mathbf u_1) \\ \vdots \\ f(\mathbf u_n)
			\end{pmatrix} \text.
		\]
	\item The $\cdot||\cdot$ operator denotes concatenation.
	\item The $((\cdot))_N$ operator takes its operand modulo $N$.
	\item Symbols $\Re$, $\Im$ denote the real and imaginary part of its argument.
	\item The symbol $\delta$ denotes the Kronecker delta.
\end{compactitem}
